<<echo=FALSE, cache=FALSE>>=
read_chunk("../code/Problem2.R")
@
\section{}
\label{sec:ex2}
From the formulations in exercise 1, a MCMC sampler can be implemented. The data used in this work is the \textit{'Oral'} dataset. 
%
<<echo=FALSE>>=
str(Oral)
attach(Oral)
@
%
The neighbourhood structure $\matr{R}$ is stored in the file 
\textit{'tma4300_ex2_Rmatrix.Rdata'}.
%
<<echo=FALSE>>=
load("./data/ex2_additionalFiles/tma4300_ex2_Rmatrix.Rdata")
@
%
In the implementation we have created a list object, \textit{'input'}, that consists of constants and the data from \textit{'Oral'}. It is used in function input in the simulation. 
%
<<echo=FALSE>>=
problem = list(
  y = Oral$Y,
  E = Oral$E,
  n = length(Oral$Y),
  R = R,
  alpha_u = 1,
  alpha_v = 1,
  beta_u = 0.01,
  beta_v = 0.01
)
@
%
%
<<r_kappa_u, eval=FALSE>>=
@
%
%
<<r_kappa_v, eval=FALSE>>=
@
%
%
<<r_u, eval=FALSE>>=
@
%
%
<<r_eta_prop, eval=FALSE>>=
@
%
%
<<d_eta, eval=FALSE>>=
@
%
%
<<r_u, eval=FALSE>>=
@
%
%
<<acceptance_prob, eval=FALSE>>=
@
%

$70000$ samples was generated and with the function \textit{system.time()} it used $923.74$ second to generate. 